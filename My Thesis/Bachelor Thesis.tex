\documentclass[12pt, a4paper, fleqn]{memoir}%makeidx

%******************************************************************************
% STYLE
%******************************************************************************
\input{style.tex}

%******************************************************************************
% BEGIN DOCUMENT
%******************************************************************************
\begin{document}

%******************************************************************************
% FRONT MATTER
%******************************************************************************
\frontmatter

%******************************************************************************
% EMPTY PAGE
%******************************************************************************
\pagestyle{empty}
This is actually the first page of the thesis and will be discarded after the print out. This is done because 
the title page has to be an even page. The memoir style package used by this template makes different indentations 
for odd and even pages which is usally done for better readability.  
\clearpage
%******************************************************************************
% TITLE PAGE
%******************************************************************************
\pagestyle{empty}
\rmfamily
\noindent
\begin{center}
University of Augsburg\\
Faculty of Applied Computer Science\\
Department of Computer Science\\
Bachelor Program in Computer Science\\
\end{center}
\begin{figure}[h]
\centering
\includegraphics[width=0.25\textwidth]{logo.png}
\end{figure}
\vfill\vfill
\begin{center}
\Large
Bachelor Thesis\\
\end{center}
\vspace{2.0em}
\begin{center}
\Large
%\LARGE Brief Title\\ \vspace{10pt} 
\Large Development of a Multi-User, Multi-Display Application to increase Energy Awareness
\end{center}
\vspace{2.0em}
\begin{center}
    \normalsize
    submitted by\\
    \large
    Karim Aly\\
    \normalsize
    on 25.08.2013
\end{center}
\vspace{2.0em}
\begin{center}
    \normalsize
    Supervisor:\\ 
    Dipl.-Inf. Michael Wissner
\end{center}
%\begin{center}
%    \normalsize
%    Adviser:\\
%    Dipl.-Inf. Michael Wi�ner
%\end{center}
\begin{center}
    \normalsize
    Reviewers:\\
    Prof. Dr. Elisabeth Andr\'{e}\\
\end{center}
\cleardoublepage

%******************************************************************************
% DEDICATION
%******************************************************************************
\vspace*{\fill}
{\hfill\sffamily\itshape}This work is dedicated for my brothers who died in recent events.
\cleardoublepage

%******************************************************************************
% ABSTRACT
%******************************************************************************
\chapter*{Abstract}
%This is the place where the \textit{abstract} of your thesis is supposed to be. The abstract is an essential part of a thesis, providing a brief summary of the thesis. Students often do not recognise the importance of the abstract and thus do not spend the required time in order to produce a well defined abstract. You should realize that the abstract is the walking advertisement for your thesis. Any reader's interest in your work stands or falls with the motivation provided by your abstract. A student should know that usually the reviewer of his or her thesis start reading with the abstract and the summary while often just making quick scans over some parts of the main chapters. An abstract is what will and has to be remembered.
Energy is now a trending topic that the whole world is talking and worried about, the energy consumption is increasing and the resources are limited so the world has to find other ways to produce energy and decrease the level of energy consumption by people. So in order to solve such a problem, the problem has to be identified. By knowing this fact and using the new modern ways as pubic displays and multi-user applications that proved to attract and motivate people to interact with it regularly in a fun and innovative way.  And that's what motivated us to create a multi-user, multi-display application to motivate people to save more energy and let them know how much energy they use.
%******************************************************************************
% ACKNOWLEDGMENTS
%******************************************************************************
\chapter*{Acknowledgments}
I would like to thank Ahmed Mohamed for helping me in understanding some concepts in the pusher service also Gasser Akila and Youssef Madkour for their support.

%******************************************************************************
% STATEMENT & DECLARATION
%******************************************************************************
\chapter*{Statement and Declaration of Consent}
\vfill
\subsubsection*{\LARGE Statement}
Hereby I confirm that this thesis is my own work and that I have documented all sources used.
\vfill
\begin{flushleft}
Karim Aly
\end{flushleft}  
\begin{flushright}
Augsburg, 25.08.2013 
\end{flushright}
\vfill
\vfill
\subsubsection*{\LARGE Declaration of Consent}
Herewith I agree that my thesis will be made available through the library of the Computer Science Department.
\vfill
\begin{flushleft}
Karim Aly
\end{flushleft}  
\begin{flushright}
Augsburg, 25.08.2013
\end{flushright}
\vfill

%******************************************************************************
% TABLE OF CONTENTS
%******************************************************************************
\cleardoublepage
\rmfamily
\normalfont
\pagenumbering{roman}
\pagestyle{headings}
\tableofcontents


%******************************************************************************
% MAIN MATTER
%******************************************************************************
\mainmatter

%##########################################################
\chapter{Introduction}
\label{chap:Introduction}

\section{Motivation}
\label{sec:Motivation}
Energy is now one of the main concerns of the world, since energy resources are running out and people's energy consumption is increasing dramatically it was expected that from 2008 to 2030 world energy consumption will increase more than 55\%\footnote{\url{http://facts.randomhistory.com/energy-facts.html}}. In addition energy consumption increases the rate of Carbon dioxide (CO2) in our environment which then leads to global warming and other environmental problems.
\\
\newline \noindent One of the comments that was online about this topic stated "It's weird how unaware we are as to how much energy we're really using"\footnote{\url{http://www.dosomething.org/actnow/tipsandtools/11-facts-about-energy}} and this is very true. People in their daily lives don't seem to know much about their energy consumption,  while cooking in the kitchen or even taking a shower or while working. So people needed anything which tells them their energy usage on a daily basis and allow them to view and compare their energy usages to hope for better energy usages.
\\
\newline \noindent Concerning public displays and multi-user applications many cities around the world now began to install public displays in many places like airports, stations, streets and many other locations since public displays have proven to motivate the users to interact with it and also help them in finding what they need in a fun and innovative way. For example the installation of interactive video wall in Copenhagen\footnote{\url{http://www.copenhagen.dk/en/whats_on/the_wall/what_is_the_wall/}}, the museum of Copenhagen decided to install a twelve meter mobile interactive video wall in the street which attracts 4000 visitors per day that talks about cultural geography and history of Copenhagen. Also the wall has a timeline toolbar which you can explore the city in different historical periods. Moreover visitors can add comments and photos and saving it on the screen and this content that the user generates can be browsed by other people as well, the wall also has a website in which users can input images from home or anyplace. In addition users can add their own photos using their mobile phones or USB.
\\
\newline \noindent Another example was in Singapore as part of Changi Millionaire campaign\footnote{\url{http://www.youtube.com/watch?v=0gl4M8NPTRg}} to further engage Singaporeans and visitors by an interactive augmented reality multimedia game at the Media Hub Wall at Orchard MRT Station. The users just stand in the specified place in front of the screen and see their images on the screen with flying virtual money in which they try to grab as much money as they could.
\\
\newline \noindent The reaction of users on both examples gave a very good impression about how users interact with public displays and how public displays applications could be the future generation of applications used in every place you go.
\\
\newline \noindent On the other hand smart phones lately have become an essential part of people's lives. People use their mobile phones now to do everything in their lives like checking e-mails, chatting with each other, taking photos and many other applications and so smart phones also were a motivation to involve them into the application.

\section{Objectives}
\label{sec:Objectives}

The aim of the project was first letting the people know how much energy they use on a daily basis because the first step to solve the problem is knowing where the problem is. To do that we decided to use modern ways like public displays,multi-user applications and smart phones as explained in the previous section in order to motivate people and engage them better in the application. We wanted to create an application which will give every detail about user's energy consumption and how to improve it, also representing this data in a fun and a new way while keeping the competitive environment on as a motivation to save more energy.
\\
\newline \noindent Moreover, helping the user in saving energy in their daily lives by giving them tips about the things they usually do every day but they don't even know that it wastes energy and giving them more efficient ways maybe this is one of the most important ways to change people's attitude towards the energy consumption case. Of course this helps if not by decreasing the energy the user consume, but by letting the user know ways about how the energy can be wasted without him knowing so and instead of doing the task in lets say a x amount of energy, it ends up using double this amount.
\\
\newline \noindent The system had to be trusted, robust and user friendly so in order to try to achieve that, privacy profiles were implemented in order to give the user an option to hide his data if someone suddenly appears or even if he doesn't want anyone to check his consumption also to increase user trust and confidence in the application. Finally we wanted to take advantage of the multi-user setup to give the users an option to view charts together in order to compare their energy consumption with each other which we believe that will motivate them to save more energy.
%\section{Outline}
%\label{sec:Outline}

%##########################################################
\chapter{Related Work}
\label{chap:RelatedWork}

\section{Theoretical Background}
\label{sec:TheoreticalBackground}
%Figure \ref{fig:intro} shows an image while you can cite a paper with \cite{AmirPnueli1985} or several papers with \cite{ThomasRist2004, Rist2002}.
Many projects were done under the topic or related somehow to the use of public displays and multi-user applications also to the energy awareness topics.
\\
\newline \noindent The first paper \cite{Article1} was  a treasure hunt game which used augmented reality technology and adaptive virtual gardens to design a game to motivate people to save energy the game consisted of two parts the first one was the augmented reality quiz consisting of some questions related to the environmental issues and there were clues at every station in the building which helped the user to go to the next one and answer the questions, this was done to increase people awareness about the energy issue. The second part was a dynamic visualisation which converted each team's score to a status of a virtual garden. So the users first start with a good health garden then according to the score they get the health of the garden differs, if they keep on getting more points the health of the garden improves. This part was concerned about motivating the users by using the gaming principles like prize, score and the factor that all the teams or users' scores where available on the public display as the garden status.
\\
\newline \noindent The second project \cite{Article2} was about a system which helps create more smart solutions and apply them to energy devices such as creating scheduled tasks for each individual device and also being able to track the number of users in a house hold and the energy consumption of each user's device to save more energy, the project mainly focuses on dealing with the problem that most of the energy monitoring devices only allow people to reduce their total energy without knowing exactly which devices consumed which amount of energy so its very hard using those systems to know how much each device or each individual is actually using.
\\
\newline \noindent So the system developed was named USEM (Ubiquitous Smart Energy Management) which aimed to assist users to save more energy without caring too much on knowing how would the system accomplish that. The system classified the devices into two categories which are regular devices and continuous devices. Regular devices were represented in devices which only take a set of time like TVs, washer and oven while continuous devices were represented in devices that operate more automatically without interference from the user like air conditioners, refrigerators and water heaters. Also the system defined rules and tasks for regular devices and defined levels for continuous devices instead of rules and tasks. So the rule was about a set of conditions, once met the rule is executed while a task is an assignment which the user schedule the system to finish it. As for the level it is also about setting a set of conditions but for continuous devices, and to be able to control and set all that from the USEM software user interface.
\\
\newline \noindent The third project \cite{Article3} was about the different interaction techniques that can be used to interact with personalized public displays in sensitive situations, mainly it compares three different interaction techniques which are direct, bodily and mobile-based. The project classified the whole process into three phases which are identification, navigation and collecting results. Also mentioned that using personalized public displays makes it easier for the user to extract his own data because it can be tailored to the user profile but on the other hand it may cause some privacy issues. The project simply explained the three main phases starting by the identification phase which is the logging in system and how the user logs in the system and his personal data is shown on the display then the navigation phase which is about the user being able to navigate through out the application to find the data he needs and finally the collecting results and logging-off phase which indicates that the user has found what he wants and whether or not the user wants to save it on his mobile then log off and the data is removed from the screen.
\\
\newline \noindent The project also discussed the three possible techniques when interacting with public displays which are direct, bodily and mobile-based, so let's take a glance at each of these techniques as discussed in the project. The direct interaction involves having an interactive tool to interact with the display which can be user's hand or NFC-enabled mobile phone, bodily interaction this technique is mainly about gestures and body postures and is usually supported by camera-based recognition and mobile-based technique which uses a mobile phone to control and interact with the public display. Each of them had their positive and negative sides also an experiment was made to test which interaction technique was preferred by the users in the three phases of interaction, the results were that users preferred mobile-based technique in Identification and Collecting results phase however users preferred direct technique in the navigation phase. Moreover it was recommended to have an auto-logout feature to increase security and trust in the system and using well known metaphors in interaction.
\\
\newline \noindent The fourth project \cite{Article4} was about how to protect and maintain user's privacy on personalized public displays and the project investigates the possible profiles that can be created for the user to choose the best privacy option for him also it investigated how the relationship context affected user's choice in deciding his privacy profile. The project classified privacy protection to five groups ordered  by ascending protection level and they are: do noting, minimize, mask, remove private part, remove all. An  Experiment was made to test which of the five privacy levels the users preferred when a stranger, colleague and a friend suddenly appears on the display and the results were that generally no protection is needed when the person appeared is a friend, the users tend to choose a more powerful privacy profile when the person is an acquaintance and finally the users tend to choose the strongest protection privacy level when the person is a stranger.

%\begin{figure}[h]
%\centering
%\includegraphics[width=0.8\textwidth]{enten.jpg}
%\caption{The map of Entenhausen}
%\label{fig:intro}
%\end{figure}

%\section{Section}
%\label{sec:Section}

%\subsection{PseudoCode}
%\label{sec:PseudoCode}
%If you want to show the implementation of some algorithm that is essential to the solution found in your thesis then do not write plain prgoram code. Use an abstract pseudocode representation instead. No one wants to see \texttt{C++\texttrademark} code or \texttt{Java\texttrademark} code in your thesis because it is presumed that you are able so write such a program as a computer scientist. Generally, writing program code is bad style and just blows up your thesis but will never be read by anyone but you. It is nothing scientific but your handwork while your thesis should show that you are able to do research as a scientist. A pseudocode example could look like the following:
%\begin{algorithm}[h]
%\caption{The Dekker Algorithm}
%\label{algo:dekker}
%\begin{algorithmic}
%\Require $n \in \mathbb{N}$
%\Require $0 \leq i,turn \leq n$
%\Require $\forall 0 \leq j \leq n : (interrested[j] = false)$
%\Procedure{DekkerAlgorithm}{$n,i$}
%  \State $interrested[i] \leftarrow true$
%  \While {$\exists 0 \leq j \leq n : (j \neq i \wedge interrested[j] = true)$}
%  \If {$turn \neq i$}
%    \State $interrested[i] \leftarrow false$
%    \While {$turn \neq i$}
%    \EndWhile
%    \State $interrested[i] \leftarrow true$
%  \EndIf 
%  \EndWhile
%  \State $ $
%  \State $\text{\textbf{\color{red}CRITICAL SECTION}}$
%  \State $ $
%  \State $turn \leftarrow Random(n)$
%  \State $interrested[i] \leftarrow false$
%\EndProcedure
%\end{algorithmic}
	%\end{algorithm}

%Be sure that each pseudocode listing is listed in the list of algorithms at the end of your thesis.

%******************************************************************************
% CONCEPT AND IMPLEMENTATION
%******************************************************************************

\chapter{Concept And Implementation}
\label{chap:ConceptAndImplementation}

\section{Technologies Used}
\label{sec:TechnologiesUsed}
The technologies used in the project were Django framework\footnote{\url{https://www.djangoproject.com/}} that was used to handle the backend code which was represented in writing the server code and creating the models or the tables of the database in order to be able to store the information needed to run the application. As for the connection between the display and the mobile, we faced a major problem at the beginning, since web sockets technology was agreed to be used but it wasn't supported by the android native mobile browsers. Another solution had to be found and that what led us to use pusher. Pusher\footnote{\url{http://pusher.com/}} is a tool which helps the developers to create applications which involves realtime in it, also phone gap\footnote{\url{http://phonegap.com/}} was used to help in writing HTML5 code and javascript to make native applications for multiple platforms so it helped in developing the backend of the mobile code which was simply javascript. For the front end code or the user interface which was represented in the design and the looks of the application, HTML5 and CSS3 were used to design the application user interface. Twitter bootstrap\footnote{\url{http://getbootstrap.com/2.3.2/}} was used also for the public display user interface with jquery\footnote{\url{http://jquery.com/}} to reach a more powerful design also Jquery mobile\footnote{\url{http://jquerymobile.com/}} was used on the mobile side for both handling the user interface and the server side functionalities on the mobile, moreover charts were needed in the application so canvasjs chart engine\footnote{\url{http://canvasjs.com/}} was used in rendering the charts with the given data using HTML5 canvas to draw the required charts.

\section{Concept}
\label{sec:Concept}
So the concept was to create an application to increase people's energy awareness and help them to save more energy. Since the technology used was agreed to be HTML5 and Web sockets, the idea was to create a multi-user application in which each user can do actions in their specified space or slot on the screen and by using web sockets the connections were established between each user(mobile) and the display. Moreover charts were created for the user to check their energy usage regularly and they could choose the duration they want to view their usage within.

%and the django framework was used as mentioned in the previous section to handle the backend code and to manage and create the models in the database in order to achieve such an application also canvasjs the javascript chart engine and pusher were used to draw user charts to illustrate their energy usage.

\section{Connection}
\label{sec:Connection}
\begin{figure}[H]
\centring
\includegraphics[width=1\textwidth]{Connection_figure.png}
\caption{This figure showing the connections made between the mobile, server and the display.}
\label{fig:Connection_figure}
\end{figure}
The connections in the project was mainly done using pusher. First the user approaches the screen to find an introduction to the application. Then the user has to click connect on his mobile device in order to start using the application.
Once the user presses connect on his mobile device a request is sent to the server which starts executing the multi-user algorithm explained below in section \ref{sec:Multi-userSetup} after finding the specified free slot on the screen it saves the slot number and sends a signal to the public display to display the login page in the specified slot using a four channel connection system which consists of a channel for each slot of the screen, when the multi-user algorithm finds the free slot it connects to the channel that corresponds to this slot (will be explained briefly in section \ref{sec:Multi-userSetup}) avoiding clashing between users' requests since each user on the display will have his own channel to send and receive requests and will also help in identifying each user actions and interactions with the system as you can see in Figure~\ref{fig:Connection_figure}.
\\
\newline \noindent As for the channels and the connections it can be seen in the following Figure~\ref{fig:ChannelsAndConnections}. By default each user on his mobile has the default channel name which is "Private-SaveE" saved then when the multi-user algorithm runs and returns the corresponding free slot number this number is concatenated to the default name "Private-SaveE" to result in the corresponding new channel name which corresponds to the slot the user will be using. The new channel names as shown in figure~\ref{fig:ChannelsAndConnections} will be one of the following: Private-SaveE1, Private-SaveE2, Private-SaveE3 and Private-SaveE4.

\begin{figure}[H]
\centring
\includegraphics[width=0.85\textwidth]{ChannelsAndConnections.png}
\caption{This figure showing the connections and the channels and how every user connects via his own channel.}
\label{fig:ChannelsAndConnections}
\end{figure}


\section{Multi-user setup}
\label{sec:Multi-userSetup}
\begin{figure}[H]
\centring
\includegraphics[width=0.80\textwidth]{SlotNumbersAndStatus.png}
\caption{The figure shows the array of slot numbers on the display.}
\label{fig:SlotNumbersAndStatus}
\end{figure}

The multi-user setup was one of the challenging milestones in the project because normally when a web application is opened in a window it will have a session for the user opening that window. However to have a number of users doing actions in the same window, here the problem occurs. Because each user is supposed to have one session id which is stored by the browser window but when multiple-users start to navigate in one window the sessions replace each other, as a result the requests origin or the user who made that request will not be known.
The problem was handled by passing the user id which is a unique key which identify a user to every page he visits once the user is logged in and neglecting the session id so doing this for all the users on the screen solves the problem of the sessions replacing each other and making sure to know where does this request comes from or from which user to avoid colliding of requests.
\\
\newline \noindent The organising and registering of a user on the display in order to start interacting with it was done by implementing a simple algorithm it's goal is to organise the order of the users and their slots on the screen to let the users interact on their specified slots by saving the number of users currently on the display in a local storage variable by the browser.
\\
\newline \noindent As you can see the following figure~\ref{fig:SlotNumbersAndStatus} represents the array of screen slots numbers from one to four when a user press the connect button the mobile sends a signal via a general channel to the server which then loops on the slot status array in figure~\ref{fig:SlotNumbersAndStatus}.
\\
\newline \noindent And when it finds a free slot which means false it overwrites the false value with a true one and take its index to get the corresponding index in the slot numbers array in figure~\ref{fig:SlotNumbersAndStatus} then it sends the slot number which will be assigned to this user to the mobile device and connect to the channel of this number and display the content on the specified slot then it increment a local storage variable stored by the browser which keep track of the number of users on the display. For example imagine this situation:
\\
\newline \noindent Currently the public display has three users logged-in and another user wants to connect so as you can see in figure~\ref{fig:Example1} the Boolean array which represents the current busy slots of the screen indicates that slot 1,2 and 3 are taken.

\begin{figure}[H]
\centring
\includegraphics[width=0.75\textwidth]{Example1.png}
\caption{The figure shows an example of users available in slot 1, 2 and 3 initially.}
\label{fig:Example1}
\end{figure}

\noindent So once the user presses connect what the system does is the following:

\newpage 
\begin{figure}[H]
\centring
\includegraphics[width=0.75\textwidth]{Example2.png}
\caption{The system in the figure iterates on the array to find an empty slot or a false value in this case.}
\label{fig:Example2}
\end{figure}

\newline \noindent The system begins to loop on the array that contains the status of each slot on the screen until it finds a cell in the array in which the cell is "false" which means at this index in the Slot number array the slot is free then the system stops and replace the status with true instead of false then sends this number to the mobile in order to connect on the channel of that slot to be able to interact with it. Figure~\ref{fig:Example3} illustrates the arrays when the algorithm executed.

\begin{figure}[H]
\centring
\includegraphics[width=0.75\textwidth]{Example3.png}
\caption{The figure shows that the system found an empty slot (false value) and will replace it with a true value to indicate that it is busy.}
\label{fig:Example3}
\end{figure}

\newline \noindent Then it sends the corresponding number in the slot numbers array which is "4" in this case to the mobile to be able to connect to slot four via its channel so as it appears in figure~\ref{fig:Example4} the number is then sent to the mobile and concatenated to the default channel name which is "Private-SaveE" as mentioned in the previous section. And by that it prevents any conflicts between users and let the system distinguish between user requests.

\begin{figure}[H]
\centring
\includegraphics[width=0.75\textwidth]{Example4.png}
\caption{The figure shows when the number of slot is sent to the mobile then the mobile concatenate that number with the default channel name to connect to the corresponding channel.}
\label{fig:Example4}
\end{figure}

\section{Synchronization}
\label{sec:Synchronization}
Synchronization of the displays was done by also using the connections as mentioned in the Connections section \ref{sec:Connection}. So the system once the user pressed any button on his mobile to navigate the system on the display the mobile pops-up the loading widget which tells the user what the server is trying to do in that moment. Then while the request reachs the server, the server sends to the display what to show then an event is triggered on the channel which the user is connected to with a loading complete header which indicates that the request has successfully executed and the loading widget is removed from the user mobile screen and the page is changed to match the display contents. Here is an example in figure~\ref{fig:Example7} to illustrate more: 

\begin{figure}[H]
\centring
\includegraphics[width=1\textwidth]{Example7.png}
\caption{This figure illustrates the synchronization process.}
\label{fig:Example7}
\end{figure}

\section{Tip Generator}
\label{sec:TipGenerator}
The tip generator was implemented to help people change their daily lives action slightly a bit in order to save much more energy so for example when cooking if the pot is not covered then definitely there is a large amount of energy lost and you need to consume more energy in order to compensate for the energy you lost when removing the pot's cover and letting heat escape. So from here came the motivation to implement the tip generator to help people to change these very small actions or attitudes towards certain situations but saves loads of energy. As you can see in figure~\ref{fig:Example10} the tip generator generates tips\footnote{\url{http://www.energymadeeasy.gov.au/energy-efficiency/saving-energy}} organised according to its topic.

\begin{figure}[H]
\centring
\includegraphics[width=1\textwidth]{Example10.png}
\caption{This figure illustrates the kind of tips generated by the tip generator.}
\label{fig:Example10}
\end{figure}

\section{Responsiveness And Interaction}
\label{sec:ResponsivenessAndInteraction}
One of the important milestones in the project were the responsiveness and the interactions since the application is a multi-user application so multiple user interface profiles had to be made in order to cover all the possible situations and combinations that might happen between users so for example when a user is on the display alone the user takes the whole screen alone to display his data but when two users are on the display, the display is splitted between them and both user interfaces for each user have to be changed in order to match the new dimensions and space that each user have on the display. So for more understanding please refer to this figure~\ref{fig:Example5} this is a situation when one user is navigating the display alone:

\begin{figure}[H]
\centring
\includegraphics[width=0.85\textwidth]{Example5.png}
\caption{The figure shows only one user navigating the display.}
\label{fig:Example5}
\end{figure}

\newline \noindent Then if another user wants to log-in and navigates, the user interface of both users change according to the number of users which is stored by the browser in the local storage as mentioned in the connection section \ref{sec:Connection} and here is a figure to illustrate more:

\begin{figure}[H]
\centring
\includegraphics[width=0.85\textwidth]{Example6.png}
\caption{The figure shows the screen when another user connected.}
\label{fig:Example6}
\end{figure}

\newline \noindent This was made by implementing css classes for each element on the screen for each possible scenario for example when a user navigates on the display alone he is taking the whole screen alone and all the elements are given the css for that situation but then when another user signs in he trigger a function which then knows the page that each user is currently at and begin to change the css of each of the elements to the new situation so the css is all checked and changed when a user connects to the display and already there are other users. Also when a user navigates from one page to another, the same function is called but to only correct the css of any change happened and to maintain the correct user interface for all the users.

\section{User Data}
\label{sec:UserData}
User data which is simply represented as charts in the project and letting the user know their energy consumption was one of the objectives to increase energy awareness and to help users save energy so user data had to be made in order to illustrate to the user his consumption also helps in two points:
\\
\newline \noindent The first point: user data helps the user to know their detailed energy consumption and as a result the user can know where does the problem come from or from which device he consumes energy the most. This part is represented as two different representations of charts the first chart is on the users profiles and shows their total energy consumption as a pie chart including each device consumption in percentages as you can see in figure~\ref{fig:Example15}. So to give the user an overview of what his total energy consumption might be the carts were placed on the profile to allow the users to check them once they login to the system since its the first thing they view once they login.

\begin{figure}[H]
\centring
\includegraphics[width=0.85\textwidth]{Example15.png}
\caption{The figure shows the pie chart displaying the total energy consumption for a user.}
\label{fig:Example15}
\end{figure}

\\
\newline \noindent The second type of charts is in the view charts section in the application its like the first type but more detailed and customized as the user wish. When users navigate to the view charts section they are asked to enter a duration in which they want to check their energy consumption within as illustrated in figure~\ref{fig:Example17}. The charts then are displayed as a pie and a bar chart, the pie chart first indicates the total energy usage in the dates entered displaying the energy consumption of each device in percentages while the bar chart displays also the energy consumption of each device but sorted or ordered monthly within also the dates the user entered. In order to get the picture more you can see figure~\ref{fig:Example16} to have an idea on how the charts are displayed.

\begin{figure}[H]
\centring
\includegraphics[width=0.35\textwidth]{Example17.png}
\caption{The figure shows how the users get to choose the dates they want to view their energy consumption within.}
\label{fig:Example17}
\end{figure}

\begin{figure}[H]
\centring
\includegraphics[width=0.85\textwidth]{Example16.png}
\caption{The figure shows the bar chart displaying in more details the energy consumption of a user sorted monthly within the period he chose.}
\label{fig:Example16}
\end{figure}

\\
\newline \noindent The second point: comparing user data with each other on the display may be another advantage of user data and the multi-user feature as users may take advantage of the multi-user setup and use the system to compare each other's data on the display for example as shown in figure~\ref{fig:Example18} two users may be standing on the display and both are viewing their energy consumption they might take advantage as mentioned and compare their data in order to motivate both of them to save energy as that can give them rewards as we will see in the game mechanics section \ref{sec:GameMechanics}.

\begin{figure}[H]
\centring
\includegraphics[width=0.75\textwidth]{Example18.png}
\caption{The figure shows two users viewing their energy consumption on the public display.}
\label{fig:Example18}
\end{figure}

\section{User Privacy}
\label{sec:UserPrivacy}
Privacy was one of the important aspects of the project since the project was not about creating an application to motivate people to save energy and increase their energy awareness only, but also a robust and trusted system, so privacy had to be implemented in the project in order to let the user feel safe at any time he doesn't want anyone to check his detailed energy consumption he just can hide his data from people and view it on his mobile screen instantaneously. So two privacy modes were created the high privacy mode which tells the system to hide the data from the public display and send the data to the mobile so the user only see his data on his mobile once he choose that option and the low privacy mode which is the normal mode, the data is kept at the display available for all users to see normally.
\\
\newline \noindent But while trying to implement privacy there were certain constraints:
\newline \noindent First the size limit of pusher service used for the connection between the public display and the mobile was only 10KB so sending the whole page to the mobile from the screen was not an option.
\newline \noindent Second the css and user interface of the page on the screen was displayed using twitter bootstrap as mentioned in section \ref{sec:TechnologiesUsed} and on the mobile Jquery mobile was used, so two different technologies used as the interface for each the display and the mobile, as a result the data couldn't be sent directly to the mobile since the styles will replace each other.
\\
\newline \noindent The two problems were solved by just sending the crucial information which was the user data in this case because there is no point in hiding the level or the money you have because it already appears on the leader board and they are made to let people compete with each other not hide it from each other so the charts engine used were put on the mobile and the data of the charts was the only part sent so by that the first problem which was the size was solved also the second one was solved because nothing is required to be sent with the data and then the mobile gives the data to the charts engine on the mobile to render it and generate the charts on the mobile once the user chooses the high privacy option. As you can see in figure~\ref{fig:Example8}, the figure shows the application with the low privacy profile setting and figure~\ref{fig:Example9} with the high privacy profile setting.
\begin{figure}[H]
\centring
\includegraphics[width=1\textwidth]{Example8.png}
\caption{This figure shows the screen and the mobile in low privacy profile setting.}
\label{fig:Example8}
\end{figure}

\begin{figure}[H]
\centring
\includegraphics[width=1\textwidth]{Example9.png}
\caption{This figure shows the screen and the mobile when the user chooses high privacy profile setting.}
\label{fig:Example9}
\end{figure}

\section{Game Mechanics}
\label{sec:GameMechanics}
Gamification or applying game mechanics is now widely used in many applications because it enhances the user experience with the application and involves more fun but in the same time it motivates users to challenge and compete against each other in order to be from the top three or the top ten players so this concept was used in the project to bring more interaction between the users and increase their motivation to compete against each other to save energy and gain badges and virtual money to be able to make their way up the leader board as it will be explained later in this section.
\newline \noindent So simply the game mechanics of the system consists of:
\\
\newline \noindent 1- Levels: level is an indication about your total consumption also used by the leader board as the sorting key.
\newline \noindent 2- Badges: a badge is gained when you cross a certain amount of money and as long as you are saving more money you gain more badges.
\newline \noindent 3- Virtual Money: initially users have money then when the costs of their usages is calculated and deducted they loose them.
\newline \noindent 4- Leader board: the leader board is the place where you can see all the users and their levels, badges, money and ranks.
\\
\newline \noindent So simply users at first have a certain amount of money by default then when they start measuring their energy consumption day by day the cost is calculated and deducted from the money they have so the more you consume energy the less the money you have and vice versa. Also when the system detects that the data is for a new month the system adds new money to the users so when the users save more energy and pay less, their money increase and when they reach a certain amount of money which is decided by the system, they gain badges which leads to a level up and improving their ranks on the leader board.
\\
\newline \noindent Now lets have a deeper look on the game mechanics within the application. First the levels are shown on the users profile page since it's the default once a user logs in the system levels are shown exactly at the top of the screen as in figure~\ref{fig:Example12}.

\begin{figure}[H]
\centring
\includegraphics[width=0.85\textwidth]{Example12.png}
\caption{This figure shows how the money and the level are displayed to the user.}
\label{fig:Example12}
\end{figure}

\newline \noindent It is placed at the top to make it easy on the user navigating the application and other users to check each other's levels. Levels is just an indication or a representation on how much money and badges you have, as long as you save more energy you earn more money and gain more badges which leads to a level up. Level is also the main key all users are sorted upon in the leader board.
\\
\newline \noindent Secondly, badges play a very important role in creating the competitive and fun environment for users to compete against each other for a certain goal. So in the application badges were used to make users eager to earn prizes when they save a certain amount of energy and motivate them to collect more badges on their profile which corresponds to saving more energy. Also badges play an important role in the level up mechanism since users level up when they pass a certain amount of money and collect more than a certain amount of badges. As you can see in figure~\ref{fig:Example13} badges are displayed on users profiles.

\begin{figure}[H]
\centring
\includegraphics[width=0.35\textwidth]{Example13.png}
\caption{This figure shows how the badges are displayed to the user.}
\label{fig:Example13}
\end{figure}

\newline \noindent Virtual money is an important factor also in the application game mechanics it represents the points that each user has which also acts as an indication on how much energy the user is saving, the higher the user's consumption the lower virtual money he has because the cost of the user consumption is deducted from the money he has. Money also is displayed at the top of the users profiles on the right of users levels as shown in figure~\ref{fig:Example12}.

\begin{figure}[H]
\centring
\includegraphics[width=0.85\textwidth]{Example14.png}
\caption{This figure shows how the leader board is shown to the users.}
\label{fig:Example14}
\end{figure}

\\
\newline \noindent Finally leader board is the place where all the users can see the level, money and badges of each other which also adds to the competitive environment in a way that all users are motivated to be from the first three rankings or in other words the top energy savers. Also it allow users to check their ranking among other users using the application just to give the users an idea on their energy consumption related to the people around them. As you can see in figure~\ref{fig:Example14} the leader board is shown with users sorted by their levels also showing their badges and money.


%******************************************************************************
% USER STUDY
%******************************************************************************

\chapter{User Study}
\label{chap:UserStudy}

\section{Participants}
\label{sec:Participants}
A study had to be made in order to test the system in many aspects some of them were usability, trust, complexity and motivation to save energy so the study was performed with ten participants the nationalities were Germans, Indians, Egyptians. Three of the  users were females while the other seven where males with age ranging from 20 years to 43 years old with an average of 27 years also their occupations were students, scientists, secretary, Phd student and a faculty member.

\section{Experiment}
\label{sec:Experiment}
The study was done by giving the users an introduction about the application and then leaving them with a list of tasks to try to accomplish them then asking them to answer questions. The questions used were the system usability scale questions with other energy saving related questions. The questions focused more on the usability, easiness and interaction of the system more than of the motivation the users got after using the system to save energy. The scale was from 1 to 5  where 1 represents strongly disagree and 5 represents strongly agree.

\newpage
\section{Results}
\label{sec:Results}
The results after analysing the data can be represented in this figure~\ref{fig:Example11}.

\begin{figure}[H]
\centring
\includegraphics[width=1\textwidth]{Example11.png}
\caption{This figure shows users impressions to the system.}
\label{fig:Example11}
\end{figure}

\\
\newline \noindent The first question was if the user had such a system fully integrated to his devices so whether he would use this system frequently or not. The average result of what the users chose was 3.8 which is almost that they agree that if there were such a system they would defiantly frequently use that system.
\\
\newline \noindent The second question involved complexity and the users were asked to tell whether or not they see the system complex. The average result was 2.1 which is nearly that the users don't agree with the concept that the system is complex.
\\
\newline \noindent The third question asked the users whether they see the system easy to use or not and the average result was 4.3 which indicates that the people agreed that it was easy to use and they didn't face any difficulties while navigating.
\\
\newline \noindent The fourth question stated whether the users faced difficulties to the extent that they would need the support of a technical person while using the system and the result had an average of 1.9 which is not agree so users didn't feel that the system is too hard that they would need the help of a technical person.
\\
\newline \noindent The fifth question asked the users if they see the different functions of the system well integrated and the average result was 4.2 which clearly says that the users had no problems with the system functions and that users saw them well integrated.
\\
\newline \noindent The six question tested the system whether there were many inconsistencies or not, results showed that the users with an average result of 1.8 disagreed that the system had any inconsistencies.
\\
\newline \noindent The seventh question tested the usability and the easiness of the system and asked users whether or not they think people would learn to use the system easily and the average result was 4.4 which indicates that the users almost strongly agreed that the system was easy to use the people would learn to use it in no time.
\\
\newline \noindent The eighth question asked users whether the system was cumbersome to use and the average result was a score of 2.1 which indicates that users disagreed that the system was cumbersome.
\\
\newline \noindent The ninth question tested users confidence in the system and the average results were 4.1, people agreed that they felt confident while using the system.
\\
\newline \noindent The tenth question asked users whether they would require to learn a lot of things before they begin to use this application and the average result was 2.5 which is between neutral and disagree.
\\
\newline \noindent The eleventh question asked the users whether the system after they've navigated and explored would such a system motivate them to save energy and the average result was 4.2 which states that users agreed that such a system would motivate them to save more energy.
\\
\newline \noindent The twelfth question asked users whether or not they found all the information they needed on the system concerning their energy usage and the average result was 4 which indicates that most of the users agreed that they found all the necessary information they needed on the system.
\\
\newline \noindent At the end the system usability scale (SUS) score had to be calculated to measure the usability of the application. SUS is considered one of the most used questionnaires to measure usability and is considered as a reliable tool for measuring usability by consisting of a ten item questionnaire and responses ranging from one to five in which one represents strongly disagree and five represents strongly agree.
\\
\newline \noindent So the SUS score for the application was calculated from the results explained above by collecting the average scores of all the participants and then for the positively-worded items (1,3,5,7 and 9) the score was the number users chose minus one and the rest of the questions (2,4,6,8 and 10) it was five minus the number the user chose then summing them all and multiplying by 2.5 at the end it gave us a SUS score of 76 so according to the graph below in figure~\ref{fig:SusScore}\footnote{\url{http://www.measuringusability.com/sus.php}} it corresponds to a B grade.

\begin{figure}[H]
\centring
\includegraphics[width=1\textwidth]{SusScore.png}
\caption{This figure shows the percentile ranks of the SUS scores.}
\label{fig:SusScore}
\end{figure}

%******************************************************************************
% CONCLUSION AND FUTURE WORK
%******************************************************************************

\chapter{Conclusion And Future Work}
\label{chap:ConclusionAndFutureWork}

\section{Conclusion}
\label{sec:Conclusion}
So at the end we wanted to create a reliable, trusted system using modern ways like public displays and smartphones to attract more people also try to solve the energy consumption issue which is considered a global problem. We were motivated by the many ideas mentioned in different papers related to the interaction of users on the display and related to interactive public displays also how to display personalized content on it. A system was built taking into consideration all the recommendations found in papers with the resources available which keep track of users energy usage and allowing them to check their energy usages regularly also it gives users tips on small things they do everyday in order to change their attitude towards this things that wastes a lot of energy. Moreover we tried to take the advantage of multi-user architecture and game mechanics in order to create a competitive environment for users to compete in gaining more virtual money and gaining badges to level up which corresponds to saving more energy and in the same time providing a privacy profiles for users to choose between them if they didn't want anyone to check their consumption at any time instantaneously.

\section{Future Work}
\label{sec:FutureWork}
As for future work on the application needs to be fully integrated with real user devices instead of dummy data also a lot of features can be added as alerts for the users when they reach a certain amount of energy consumption for a certain device or for their total energy usage that the user themselves set. Also replacing the online service pusher which gives you the option to put realtime events in you application by an offline service so the user doesn't need to access the internet and can use the application while offline.
\\
\newline \noindent Moreover implementing a semaphore-like mechanism to prevent the shared variables which in this case is the number of users stored by the browser which can cause a problem in knowing the actual number of users and depending on it the layout could be broken also while connecting users manipulate a shared variable which is represented in the default channel all the users connects through it to the server before their data is displayed on the screen if two or more users connects at the same exact time a problem occurs and a conflict between requests occurs.
\\
\newline \noindent Also creating a more responsive layout in order to help users feel more comfortable with the system maybe with more animations since from my point of view user interface is one of the most important factors that affects any product nowadays.

%******************************************************************************
% BIBLIOGRAPHY
%******************************************************************************
\bibliographystyle{plain}
{\small\bibliography{BachelorThesis}}

%******************************************************************************
% APPENDIX
%******************************************************************************
\appendix
\appendixpage*
\chapter{Questionnaire}
\label{app:Questionnaire}
This is the introduction, tasks and questionnaires that was given to the users to evaluate the system.
\\
\newline \noindent Introduction:
\\
\newline \noindent This application helps you to keep track of your energy consumption and compete with other users to save more energy.
\\
\newline \noindent The application mainly allows you to view charts of your energy usage and gain badges and virtual money as long as you save more energy. Also you can check your rank among other users too. Moreover you can hide your personal data to avoid anyone to see it.
\\
\newline \noindent Notes:
\newline 1- Assume the data that you will see is your energy usage for 4 months.
\newline 2- The system will log you out after 5 minutes if you are idle
\\
\newline \noindent Users:
\newline 1- Benutzername: Bob Smith
\newline Kennwort: b

\begin{figure}[h]
\centring
\includegraphics[width=0.55\textwidth]{Example19.png}
\caption{This figure illustrates the buttons and main features in the application.}
\label{fig:Intro}
\end{figure}

\newpage
\newline \noindent Here you can find the tasks that were given to the users to accomplish.
\\
\newline \noindent \textbf{Tasks}:
\\
\newline \indent 1. Login.
\\
\newline \indent 2. State your level.
\\
\newline \indent 3. State the amount of money you have.
\\
\newline \indent 4. State the badges that you've won.
\\
\newline \indent 5. State the tip of the day.
\\
\newline \indent 6. State your rank on the leader board.
\\
\newline \indent 7. View your total energy usage:
\newline \indent \indent - State in (\%) the amount of energy your display used in total.
\\
\newline \indent 8. View your energy usage from 1-August to 1-October:
\newline \indent \indent - State in (\%) the amount of energy your CPU used in total.
\\
\newline \indent 9. Assume that a stranger suddenly appeared try to hide your data.
\\
\newline \indent 10.Logout.

\newpage
\\

\onehalfspacing


\def\MBox{\Large\ensuremath{\Box}}

\newcolumntype{S}{>{\centering\arraybackslash}m{3.0em}}
\newcolumntype{X}{>{\centering\arraybackslash}m{1.0em}}

\renewcommand{\tabularxcolumn}[1]{m{#1}} % redefine 'X' to use 'm'
%%%%%%%%%%%%%%%%%%%%%%%%%%%%%%%%%%%%%%%%%%%%%%%%%%%%%%%%%%%%
%% Beginning of questionnaire command definitions         %%
%%%%%%%%%%%%%%%%%%%%%%%%%%%%%%%%%%%%%%%%%%%%%%%%%%%%%%%%%%%%
%%
%% 2010 by Sven Hartenstein
%% mail@svenhartenstein.de
%% http://www.svenhartenstein.de
%%
%% Please be warned that this is NOT a full-featured framework for
%% creating (all sorts of) questionnaires. Rather, it is a small
%% collection of LaTeX commands that I found useful when creating a
%% questionnaire. Feel free to copy and adjust any parts you like.
%% Most probably, you will want to change the commands, so that they
%% fit your taste.
%%
%% Also note that I am not a LaTeX expert! Things can very likely be
%% done much more elegant than I was able to. If you have suggestions
%% about what can be done better please send me an email. I intend to
%% add good tipps to my website and to name contributers of course.

%% \Qq = Questionaire question. Oh, this is just too simple. It helps
%% making it easy to globally change the appearance of questions.
\newcommand{\Qq}[1]{\textbf{#1}}

%% \QO = Circle or box to be ticked. Used both by direct call and by
%% \Qrating and \Qlist.
\newcommand{\QO}{$\ocircle$}% or: $\ocircle$

%% \Qrating = Automatically create a rating scale with NUM steps, like
%% this: 0--0--0--0--0.
\newcounter{qr}
\newcommand{\Qrating}[1]{\QO\forloop{qr}{1}{\value{qr} < #1}{---\QO}}

%% \Qline = Again, this is very simple. It helps setting the line
%% thickness globally. Used both by direct call and by \Qlines.
\newcommand{\Qline}[1]{\rule{#1}{0.6pt}}

%% \Qlines = Insert NUM lines with width=\linewith. You can change the
%% \vskip value to adjust the spacing.
\newcounter{ql}
\newcommand{\Qlines}[1]{\forloop{ql}{0}{\value{ql}<#1}{\vskip0em\Qline{\linewidth}}}

%% \Qlist = This is an environment very similar to itemize but with
%% \QO in front of each list item. Useful for classical multiple
%% choice. Change leftmargin and topsep accourding to your taste.
\newenvironment{Qlist}{%
  \renewcommand{\labelitemi}{\QO}
  \begin{itemize}[leftmargin=1.5em,topsep=-.5em]
}{%
  \end{itemize}
}

%% \Qtab = A "tabulator simulation". The first argument is the
%% distance from the left margin. The second argument is content which
%% is indented within the current row.
\newlength{\qt}
\newcommand{\Qtab}[2]{
  \setlength{\qt}{\linewidth}
  \addtolength{\qt}{-#1}
%  \hfill\parbox[t]{\qt}{\raggedright #2}
\parbox[t]{\qt}{\raggedright #2}

}

%% \Qitem = Item with automatic numbering. The first optional argument
%% can be used to create sub-items like 2a, 2b, 2c, ... The item
%% number is increased if the first argument is omitted or equals 'a'.
%% You will have to adjust this if you prefer a different numbering
%% scheme. Adjust topsep and leftmargin as needed.
\newcounter{itemnummer}
\newcommand{\Qitem}[2][]{% #1 optional, #2 notwendig
  \ifthenelse{\equal{#1}{}}{\stepcounter{itemnummer}}{}
  \ifthenelse{\equal{#1}{a}}{\stepcounter{itemnummer}}{}
  \begin{enumerate}[topsep=2pt,leftmargin=2.8em]
  \item[\textbf{\arabic{itemnummer}#1.}] #2
  \end{enumerate}
}

%% \QItem = Like \Qitem but with alternating background color. This
%% might be error prone as I hard-coded some lengths (-5.25pt and
%% -3pt)! I do not yet understand why I need them.
\definecolor{bgodd}{rgb}{0.9,0.9,0.9}
\definecolor{bgeven}{rgb}{0.95,0.95,0.95}
\newcounter{itemoddeven}
\newlength{\gb}
\newcommand{\QItem}[2][]{% #1 optional, #2 notwendig
  \setlength{\gb}{\linewidth}
  %\setlength{\gb}{19cm}
  \addtolength{\gb}{-5.25pt}
  
  %\addtolength{\gb}{-5.25pt}
  \ifthenelse{\equal{\value{itemoddeven}}{0}}{%
    \noindent\colorbox{bgeven}{\hskip-3pt\begin{minipage}{\gb}\bigskip\Qitem[#1]{#2}\vspace{5 mm}\end{minipage}}%
    \stepcounter{itemoddeven}%
    
  }{%
    \noindent\colorbox{bgodd}{\hskip-3pt\begin{minipage}{\gb}\bigskip\Qitem[#1]{#2}\vspace{5 mm}\end{minipage}}%
    \setcounter{itemoddeven}{0}%
    
  }
}


%%%%%%%%%%%%%%%%%%%%%%%%%%%%%%%%%%%%%%%%%%%%%%%%%%%%%%%%%%%%
%% End of questionnaire command definitions               %%
%%%%%%%%%%%%%%%%%%%%%%%%%%%%%%%%%%%%%%%%%%%%%%%%%%%%%%%%%%%%
\newline \noindent The following is the System Usability Scale questionnaire in addition to questions related to the motivation to save energy.
\\
\\ 
\makebox[\textwidth]{%
\begin{tabularx}{17cm}{ X S S S S S S}
\textbf{Questionnaire:}
\\
\\
   &  & \footnotesize{Strongly disagree} &  &  &  & \footnotesize{Strongly agree}\\
   &  & 1 & 2 & 3 & 4 & 5\\
  %\aline
\textbf{1. I think that I would like to use this system frequently.} &  & \MBox & \MBox & \MBox & \MBox & \MBox\\
%\hline
 & & & & & & \\
\textbf{2. I found the system unnecessarily complex.} &  & \MBox & \MBox & \MBox & \MBox & \MBox\\
 & & & & & & \\
\textbf{3. I thought the system was easy to use.} &  & \MBox & \MBox & \MBox & \MBox & \MBox\\
 & & & & & & \\
\textbf{4. I think that I would need the support of a technical person to be able to use this system.} &  & \MBox & \MBox & \MBox & \MBox & \MBox\\ & & & & & & \\
\textbf{5. I found the various functions in this system were well integrated.} &  & \MBox & \MBox & \MBox & \MBox & \MBox\\ & & & & & & \\
\textbf{6. I thought there was too much inconsistency in this system.} &  & \MBox & \MBox & \MBox & \MBox & \MBox\\ & & & & & & \\
\textbf{7. I would imagine that most people would learn to use this system very quickly.} &  & \MBox & \MBox & \MBox & \MBox & \MBox\\ & & & & & & \\
\end{tabularx}%
}

\newpage

\makebox[\textwidth]{%
\begin{tabularx}{17cm}{ X S S S S S S}
\textbf{8. I found the system very cumbersome to use.} &  & \MBox & \MBox & \MBox & \MBox & \MBox\\ & & & & & & \\
\textbf{9. I felt very confident using the system.} &  & \MBox & \MBox & \MBox & \MBox & \MBox\\ & & & & & & \\
\textbf{10. I needed to learn a lot of things before I could get going with this system.} &  & \MBox & \MBox & \MBox & \MBox & \MBox\\& & & & & & \\
\textbf{11. The system would motivate me to save more energy.} &  & \MBox & \MBox & \MBox & \MBox & \MBox\\& & & & & & \\
\textbf{12. The system provided all the information i needed on my energy usage.} &  & \MBox & \MBox & \MBox & \MBox & \MBox\\

\end{tabularx}%
}
%This is the place where the appendices are supposed to be. Appendices are everything that would just blow up your thesis but are still of some interest for a reader that wants to get a deeper grasp on the details of your work.


%******************************************************************************
% BACK MATTER
%******************************************************************************
\backmatter

%******************************************************************************
% LIST OF SYMBOLS
%******************************************************************************
%\normalfont
%\clearpage
%\chapter[List of Symbols and Abbreviations]{List of Symbols and Abbreviations}
%\begin{center}
%\small
%\begin{longtable}{lp{3.0in}c}
%\toprule
%\multicolumn{1}{c}{Abbreviation} & \multicolumn{1}{c}{Description}\\ \midrule\addlinespace[2pt] \endhead
%\bottomrule\endfoot
%XML & E\textbf{X}tensible \textbf{M}arkup \textbf{L}anguage \\
%XSD & \textbf{X}ML-\textbf{S}chema-\textbf{D}efinition \\
%SFXML & \textbf{S}cene\textbf{F}low E\textbf{X}tensible \textbf{M}arkup \textbf{L}anguage \\
%SFTXL & \textbf{S}cene\textbf{F}low \textbf{T}extual E\textbf{X}pression \textbf{L}anguage \\
%SCXML & \textbf{S}tate\textbf{C}hart E\textbf{X}tensible \textbf{M}arkup \textbf{L}anguage \\
%DOM & \textbf{D}ocument \textbf{O}bject \textbf{M}odel \\
%LR & \textbf{L}eft to \textbf{R}ightmost derivation \\
%LALR & \textbf{L}ook\textbf{A}head LR\\
%NPC & \textbf{N}on-\textbf{P}erson-\textbf{C}haracter\\
%ABL & \textbf{A} \textbf{B}ehavior \textbf{L}anguage\\
%\end{longtable}
%\end{center}

%******************************************************************************
% LIST OF FIGURES
%******************************************************************************
\normalfont
\clearpage
\listoffigures

%******************************************************************************
% LIST OF TABLES
%******************************************************************************
\normalfont
\clearpage
\listoftables

%******************************************************************************
% LIST OF ALGORITHMS
%******************************************************************************
%\normalfont
\clearpage
\listofalgorithms

%******************************************************************************
% END DOCUMENT
%******************************************************************************
\end{document}
