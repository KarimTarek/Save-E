\documentclass[12pt, a4paper, fleqn]{memoir}%makeidx

%******************************************************************************
% STYLE
%******************************************************************************
\input{style.tex}

%******************************************************************************
% BEGIN DOCUMENT
%******************************************************************************
\begin{document}

%******************************************************************************
% FRONT MATTER
%******************************************************************************
\frontmatter

%******************************************************************************
% EMPTY PAGE
%******************************************************************************
\pagestyle{empty}
This is actually the first page of the thesis and will be discarded after the print out. This is done because 
the title page has to be an even page. The memoir style package used by this template makes different indentations 
for odd and even pages which is usally done for better readability.  
\clearpage
%******************************************************************************
% TITLE PAGE
%******************************************************************************
\pagestyle{empty}
\rmfamily
\noindent
\begin{center}
University of Augsburg\\
Faculty of Applied Computer Science\\
Department of Computer Science\\
Bachelor Program in Computer Science\\
\end{center}
\begin{figure}[h]
\centering
\includegraphics[width=0.25\textwidth]{logo.png}
\end{figure}
\vfill\vfill
\begin{center}
\Large
Bachelor Thesis\\
\end{center}
\vspace{2.0em}
\begin{center}
\Large
\LARGE Brief Title\\ \vspace{10pt} 
\Large Development of a Multi-User, Multi-Display application to increase Energy Awareness
\end{center}
\vspace{2.0em}
\begin{center}
    \normalsize
    submitted by\\
    \large
    Karim Aly\\
    \normalsize
    on 25.08.2013
\end{center}
\vspace{2.0em}
\begin{center}
    \normalsize
    Supervisor:\\ 
    Prof. Dr. Elisabeth Andr\'{e}
\end{center}
\begin{center}
    \normalsize
    Adviser:\\
    Dipl.-Inf. Michael Wi�ner
\end{center}
\begin{center}
    \normalsize
    Reviewers:\\
    Prof. Dr. Elisabeth Andr\'{e}\\
    Prof. Dr. Elisabeth Andr\'{e}\\
\end{center}
\cleardoublepage

%******************************************************************************
% DEDICATION
%******************************************************************************
\vspace*{\fill}
{\hfill\sffamily\itshape} I want to dedicate this work for my parents who helped and supported me along the way
\cleardoublepage

%******************************************************************************
% ABSTRACT
%******************************************************************************
\chapter*{Abstract}
%This is the place where the \textit{abstract} of your thesis is supposed to be. The abstract is an essential part of a thesis, providing a brief summary of the thesis. Students often do not recognise the importance of the abstract and thus do not spend the required time in order to produce a well defined abstract. You should realize that the abstract is the walking advertisement for your thesis. Any reader's interest in your work stands or falls with the motivation provided by your abstract. A student should know that usually the reviewer of his or her thesis start reading with the abstract and the summary while often just making quick scans over some parts of the main chapters. An abstract is what will and has to be remembered.
Energy is now a trending issue that the whole world is talking and is worried about , the energy consumption is increasing and the resources are limited so the world has to find other ways to produce energy and decrease the level of energy consumption by people and in order to solve such a problem, the problem has to be identified.
So by knowing this fact and using the new modern ways as pubic displays and multi-user applications which was proven to motivate people to interact with it regularly in a fun and innovative way and that's how the idea of creating a multi-user, multi-display application to motivate people to save more energy and let them know how much energy they use came. 
%******************************************************************************
% ACKNOWLEDGMENTS
%******************************************************************************
\chapter*{Acknowledgments}
I would like to thank Ahmed Mohamed for helping me in understanding some concepts in the pusher service also Gasser Akila and Youssef Madkour for their support.

%******************************************************************************
% STATEMENT & DECLARATION
%******************************************************************************
\chapter*{Statement and Declaration of Consent}
\vfill
\subsubsection*{\LARGE Statement}
Hereby I confirm that this thesis is my own work and that I have documented all sources used.
\vfill
\begin{flushleft}
Karim Aly
\end{flushleft}  
\begin{flushright}
Augsburg, 25.08.2013 
\end{flushright}
\vfill
\vfill
\subsubsection*{\LARGE Declaration of Consent}
Herewith I agree that my thesis will be made available through the library of the Computer Science Department.
\vfill
\begin{flushleft}
Karim Aly
\end{flushleft}  
\begin{flushright}
Augsburg, 25.08.2013
\end{flushright}
\vfill

%******************************************************************************
% TABLE OF CONTENTS
%******************************************************************************
\cleardoublepage
\rmfamily
\normalfont
\pagenumbering{roman}
\pagestyle{headings}
\tableofcontents


%******************************************************************************
% MAIN MATTER
%******************************************************************************
\mainmatter

%##########################################################
\chapter{Introduction}
\label{chap:Introduction}

\section{Motivation}
\label{sec:Motivation}

\section{Objectives}
\label{sec:Objectives}

%\section{Outline}
%\label{sec:Outline}

%##########################################################
\chapter{Related Work}
\label{chap:RelatedWork}

\section{Theoretical Background}
\label{sec:TheoreticalBackground}
%Figure \ref{fig:intro} shows an image while you can cite a paper with \cite{AmirPnueli1985} or several papers with \cite{ThomasRist2004, Rist2002}.

%\begin{figure}[h]
%\centering
%\includegraphics[width=0.8\textwidth]{enten.jpg}
%\caption{The map of Entenhausen}
%\label{fig:intro}
%\end{figure}

%\section{Section}
%\label{sec:Section}

%\subsection{PseudoCode}
%\label{sec:PseudoCode}
%If you want to show the implementation of some algorithm that is essential to the solution found in your thesis then do not write plain prgoram code. Use an abstract pseudocode representation instead. No one wants to see \texttt{C++\texttrademark} code or \texttt{Java\texttrademark} code in your thesis because it is presumed that you are able so write such a program as a computer scientist. Generally, writing program code is bad style and just blows up your thesis but will never be read by anyone but you. It is nothing scientific but your handwork while your thesis should show that you are able to do research as a scientist. A pseudocode example could look like the following:
%\begin{algorithm}[h]
%\caption{The Dekker Algorithm}
%\label{algo:dekker}
%\begin{algorithmic}
%\Require $n \in \mathbb{N}$
%\Require $0 \leq i,turn \leq n$
%\Require $\forall 0 \leq j \leq n : (interrested[j] = false)$
%\Procedure{DekkerAlgorithm}{$n,i$}
%  \State $interrested[i] \leftarrow true$
%  \While {$\exists 0 \leq j \leq n : (j \neq i \wedge interrested[j] = true)$}
%  \If {$turn \neq i$}
%    \State $interrested[i] \leftarrow false$
%    \While {$turn \neq i$}
%    \EndWhile
%    \State $interrested[i] \leftarrow true$
%  \EndIf 
%  \EndWhile
%  \State $ $
%  \State $\text{\textbf{\color{red}CRITICAL SECTION}}$
%  \State $ $
%  \State $turn \leftarrow Random(n)$
%  \State $interrested[i] \leftarrow false$
%\EndProcedure
%\end{algorithmic}
	%\end{algorithm}

%Be sure that each pseudocode listing is listed in the list of algorithms at the end of your thesis.

%******************************************************************************
% CONCEPT AND IMPLEMENTATION
%******************************************************************************

\chapter{Concept And Implementation}
\label{chap:ConceptAndImplementation}

\section{Technologies Used}
\label{sec:TechnologiesUsed}
The technologies used in the project were Django framework that was used to handle the backend code which was represented in writing the server code and creating the models or the tables of the database in order to be able to store the information needed to run the application as for the connection between the display and the mobile, we faced a major problem at the beginning, since web sockets technology was agreed to be used and it wasn't supported by the android native mobile browsers another solution had to be found and that what led us to pusher. Pusher is a tool which helps the developers to create applications which involves realtime in it, also phone gap was used to help in writing HTML5 code and javascript to make native applications for multiple platforms so it helped in developing the backend of the mobile code which was simply javascript. For the front end code or the user interface which was represented in the design and the looks of the application, HTML5 and CSS3 were used to design the application user interface and twitter bootstrap was used also for the public display user interface with jquery to reach a more powerful design also Jquery mobile was used on the mobile side for both handling the user interface and the server side functionalities on the mobile, moreover charts were needed in the application so canvasjs chart engine was used in rendering the charts with the given data using HTML5 canvas to draw the required charts.

\section{Concept}
\label{sec:Concept}
So the concept was to create an application to help people to increase their energy awareness and help them to save more energy. Since the technology used was agreed to be HTML5 and Web sockets, the idea was to create a multi-user application in which each user can do actions in their specified space or slot on the screen and by using web sockets the connection were established between each user(mobile) and the display. Moreover charts is created for the user to check regularly to check their energy usage and they get to choose the duration they want to view their usage into.

%and the django framework was used as mentioned in the previous section to handle the backend code and to manage and create the models in the database in order to achieve such an application also canvasjs the javascript chart engine and pusher were used to draw user charts to illustrate their energy usage.

\section{Connection}
\label{sec:Connection}
\begin{figure}[h]
\centring
\includegraphics[width=1\textwidth]{Connection_figure.png}
\caption{This figure showing the connections made between the mobile,server and the display}
\label{fig:Connection_figure}
\end{figure}
The connections in the project was mainly done using pusher first the user approaches the screen to find an introduction to the application and that the user has to click connect in order to start using the application.
Once the user press connect on his mobile device a request is sent to the server which start executing the multi-user algorithm explained below in section \ref{sec:Multi-userSetup} after finding the specified free slot on the screen it saves that slot number and send a signal to the public display to display the login page in the specified slot which the user can login and start navigating the application as you can see in Figure~\ref{fig:Connection_figure}.

\section{Multi-user setup}
\label{sec:Multi-userSetup}
The multi-user setup was one of the challenging part in the project because normally when a web application is opened in a window it will have a session for the user opening that window but to have a number of users doing some actions in the same window here the problem occurs because each user is supposed to have one session but when multiple-users start to navigate in one window the sessions replace each other as a result the requests origin or the user who made that request is unknown.
The problem was handled by passing the user id which is a unique key which identify a user to every page he visits once the user is logged in and neglecting the session id so doing this for all the user on the screen solves the problem of the sessions replacing each other and making sure to know what does this request comes from or from which user to be specific.


%******************************************************************************
% CONCLUSION
%******************************************************************************

\chapter{Conclusion}
\label{chap:Conclusion}

%******************************************************************************
% RESULTS AND FUTURE WORK
%******************************************************************************

\chapter{Results And Future Work}
\label{chap:ResultsAndFutureWork}


%******************************************************************************
% BIBLIOGRAPHY
%******************************************************************************



%******************************************************************************
% BIBLIOGRAPHY
%******************************************************************************
\bibliographystyle{plain}
{\small\bibliography{master}}

%******************************************************************************
% APPENDIX
%******************************************************************************
\appendix
\appendixpage*
\chapter{First Appendix}
\label{app:FirstAppendix}
This is the place where the appendices are supposed to be. Appendices are everything that would just blow up your thesis but are still of some interrest for a reader that wants to get a deeper grasp on the details of your work.

%******************************************************************************
% BACK MATTER
%******************************************************************************
\backmatter

%******************************************************************************
% LIST OF SYMBOLS
%******************************************************************************
%\normalfont
%\clearpage
%\chapter[List of Symbols and Abbreviations]{List of Symbols and Abbreviations}
%\begin{center}
%\small
%\begin{longtable}{lp{3.0in}c}
%\toprule
%\multicolumn{1}{c}{Abbreviation} & \multicolumn{1}{c}{Description}\\ \midrule\addlinespace[2pt] \endhead
%\bottomrule\endfoot
%XML & E\textbf{X}tensible \textbf{M}arkup \textbf{L}anguage \\
%XSD & \textbf{X}ML-\textbf{S}chema-\textbf{D}efinition \\
%SFXML & \textbf{S}cene\textbf{F}low E\textbf{X}tensible \textbf{M}arkup \textbf{L}anguage \\
%SFTXL & \textbf{S}cene\textbf{F}low \textbf{T}extual E\textbf{X}pression \textbf{L}anguage \\
%SCXML & \textbf{S}tate\textbf{C}hart E\textbf{X}tensible \textbf{M}arkup \textbf{L}anguage \\
%DOM & \textbf{D}ocument \textbf{O}bject \textbf{M}odel \\
%LR & \textbf{L}eft to \textbf{R}ightmost derivation \\
%LALR & \textbf{L}ook\textbf{A}head LR\\
%NPC & \textbf{N}on-\textbf{P}erson-\textbf{C}haracter\\
%ABL & \textbf{A} \textbf{B}ehavior \textbf{L}anguage\\
%\end{longtable}
%\end{center}

%******************************************************************************
% LIST OF FIGURES
%******************************************************************************
\normalfont
\clearpage
\listoffigures

%******************************************************************************
% LIST OF TABLES
%******************************************************************************
\normalfont
\clearpage
\listoftables

%******************************************************************************
% LIST OF ALGORITHMS
%******************************************************************************
%\normalfont
\clearpage
\listofalgorithms

%******************************************************************************
% END DOCUMENT
%******************************************************************************
\end{document}
